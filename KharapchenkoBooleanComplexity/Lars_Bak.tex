\section{Motivation}
\begin{frame}{Forschung seit 1971}
    \begin{block}{Elias Koutsoupias: Improvements on Khrapchenko's theorem (1993)}
    \begin{columns}[onlytextwidth,T]
      \column{\dimexpr\linewidth-30mm-5mm}
        \say{[...] we know of no Boolean function where our method improves upon Kharapchenko's theorem by a factor \emph{larger than two}, when $A,B$ are chosen appropriately}\\\vspace*{1cm}
        Somit $O(n^2)$!
      \column{30mm}
      \includegraphics[width=30mm]{Koutsoupias.jpg}
      Quelle: \url{cs.ox.ac.uk}
      \end{columns}
    \end{block}
\end{frame}

\begin{frame}{Weitere Forschung seit 1971}
    \begin{block}{S. Laplante, T. Lee, M. Szegedy: The Quantum Adversary Method and Classical Formula Size Lower Bounds (2006)}
        Zu einem Optimierungsansatz von Karchmer (1995):\\~\\
        \say{They show that this bound is larger than the bound given by Kharapchenko's method, but cannot prove lower bounds larger than $n^2$}\\~\\
        \pause
        Wieso ist die $n^2$ Schranke so schwierig zu brechen?
    \end{block}
\end{frame}



\section{Generalisierte Kostenma\ss{}e}

\begin{frame}{}
    \begin{block}{Definition Komplexit\"atsma\ss{}}
        Sei $\mathcal{F}_n$ die Menge aller boolschen Funktionen der Form 
        \[
        f : \{0,1\}^n \rightarrow \{0,1\}
        \]
        Eine Funktion $\nu : \mathcal{F}_n \rightarrow \mathbb{R}$ heisst \textit{formles Komplexit\"atsma\ss{}}, wenn
        \pause
        \begin{itemize}
            \item[(a)]Das Ma\ss{} jeden Literals $\leq 1$ ist (Normalisierung)
            \pause
            \item[(b)]Fuer alle $g,h \in \mathcal{F}_n$ gilt, dass
            \[
                \nu(g \lor h) \leq \nu(g) + \nu(h)
            \]
            (Subadditivit\"at)
        \end{itemize}
    \end{block}
\end{frame}

\begin{frame}{}
    \begin{block}{Subadditives Rechtecksma\ss{}}
        $\mathcal{R}(S)$: Menge aller Unterrechtecke von $S$.\\\vspace*{0.5cm}
        \pause
        $\mu : \mathcal{R}(S) \rightarrow \mathbb{R}$ \textit{(stark) subadditives Rechtecksma\ss{}}, wenn
        \begin{itemize}
            \item[(i)] (Normalisierung) $\mu(M) \leq 1$, $M$ monochrom
            \item[(ii)] (Subadditivit\"at) F\"ur disjunkte Partition $R = R_1 \cup \dots \cup R_m$
            \[
                \mu(R) \leq \sum_{i=1}^m \mu(R_i)
            \]
            \pause
        \end{itemize}
        \begin{itemize}
            \item \textbf{Subadditivit\"at}: $m = 2$\\
            \item \textbf{Starke Subadditivit\"at}: $m$ beliebig
        \end{itemize}
    \end{block}
\end{frame}

\begin{frame}{}
    \begin{block}{Gebrochene Partitionen}
        Seien 
        \begin{itemize}
            \item $R$ ein Rechteck
            \item $R_1,\dots,R_m$ dessen Unterrechtecke mit
            \item Gewichten $r_1,\dots,r_m \in [0,1]$
        \end{itemize}
        \pause
        $R_1,\dots,R_m$ bilden eine \textit{gebrochene Partition}, wenn f\"ur alle Kanten $e$
        \[
            \sum_{i : e \in R_i} r_i = 1
        \]
        gilt.
    \end{block}
\end{frame}

\begin{frame}{}
    \begin{block}{Konvexit\"at}
        Sei $\mu$ stark subadditives Rechtecksma\ss{}.\\~\\
        \pause
        $\mu$ ist \emph{konvex}, wenn f\"ur beliebige gebrochene Partitionen $S = \sum_{i=1}^m r_i R_i$
        \[
            \mu(S) \leq \sum_{i=1}^m r_i \mu(R_i)
        \]
        gilt.
    \end{block}
\end{frame}

\begin{frame}{}
    \begin{block}{\"Uberblick Ma\ss{}e}
        \begin{itemize}
            \item Subadditiv: 
            \[
                \mu(R) \leq \mu(R_1) + \mu(R_2)
            \]
            \pause
            \item Stark Subadditiv: 
            \[
                \mu(R) \leq \sum_{i=1}^m \mu(R_i)
            \]
            \pause
            \item Konvex: \[\mu(R) \leq \sum_{i=1}^m r_i \mu(R_i)\]
        \end{itemize}
    \end{block}
\end{frame}

\begin{frame}{}
    \begin{block}{Gebrochene Partitionsnummer}
    Die gebrochene Partitionsnummer ist das Minimum
    \[
        \pi(S) := \min \sum_{i=1}^t r_i
    \]
    \"uber alle \textbf{monochromen} gebrochenen Partitionen $R_1,\dots,R_t$ und Gewichte $r_1,\dots,r_t$.\\
    \pause
    Hierbei handelt es sich um die gewichtete Version von $\chi(S)$
    \end{block}
\end{frame}

\begin{frame}{}
    \begin{block}{Lemma subadditive Rechtecksma\ss{}e}
        $\chi(R)$ ist das gr\"o\ss{}te stark subadditive Rechtecksma\ss{}, d.h.
        \begin{itemize}
            \item $\chi(R)$ ist stark subadditiv
            \item $\mu(R) \leq \chi(R)$
        \end{itemize}
        f\"ur alle Rechtecke $R$ und stark subadditiven Rechtecksma\ss{}en $\mu$.
    \end{block}
    \pause
    \begin{block}{Lemma konvexe Rechtecksma\ss{}e}
        $\pi(R)$ ist das gr\"o\ss{}te konvexe Rechtecksma\ss{}, d.h.
        \begin{itemize}
            \item $\pi(R)$ ist konvex
            \item $\mu(R) \leq \pi(R)$
        \end{itemize}
        f\"ur alle Rechtecke $R$ und konvexe Rechtecksma\ss{}en $\mu$.
    \end{block}
\end{frame}

% Hier ist beabsichtigt, dass nur das zweite Lemma bewiesen wird.
\begin{frame}[t]{Beweis konvexe Rechtecksma\ss{}e}
    Zu zeigen: $\pi$ ist konvex.
    \pause
    Seien
    \begin{itemize}
        \item $S = \sum_{j} r_j$ \colorbox{dominant_blue}{\strut$R_j$} eine gebrochene Partition
        \item \colorbox{dominant_blue}{\strut$R_j = \sum_{i} s_{ij} M_{ij}$} gebrochene Partitionen f\"ur alle $j$, so dass
        \begin{itemize}
            \item $M_{ij}$ monochrom
            \item $\pi(R_j) = \sum_i s_{ij}$ (sprich minimal) 
        \end{itemize}
    \end{itemize}
\end{frame}

\begin{frame}[t,noframenumbering]{Beweis konvexe Rechtecksma\ss{}e}
    Zu zeigen: $\pi$ ist konvex.
    Seien
    \begin{itemize}
        \item $S = \sum_{j} $ \colorbox{roman_silver}{\strut$r_j$} $R_j$ eine gebrochene Partition
        \item \colorbox{dominant_blue}{\strut$R_j = \sum_{i} s_{ij} M_{ij}$} gebrochene Partitionen f\"ur alle $j$, so dass
        \begin{itemize}
            \item $M_{ij}$ monochrom
            \item $\pi(R_j) = \sum_i s_{ij}$ (sprich minimal) 
        \end{itemize}
    \end{itemize}
    Dies kann man umformen zu 
    \[
        S = \sum_{ij} \colorbox{roman_silver}{\strut$r_j$} \colorbox{dominant_blue}{\strut$s_{ij} M_{ij}$}
    \]
\end{frame}

\begin{frame}[t,noframenumbering]{Beweis konvexe Rechtecksma\ss{}e}
    Zu zeigen: $\pi$ ist konvex.
    Seien
    \begin{itemize}
        \item $S = \sum_{j} r_j R_j$ eine gebrochene Partition
        \item $R_j = \sum_{i} s_{ij} M_{ij}$ gebrochene Partitionen f\"ur alle $j$, so dass
        \begin{itemize}
            \item $M_{ij}$ monochrom
            \item $\pi(R_j) = \sum_i s_{ij}$ (sprich minimal) 
        \end{itemize}
    \end{itemize}
    Dies kann man umformen zu 
    \[
        S = \sum_{ij} r_j s_{ij} M_{ij}
    \]
    Da $\pi$ das Minimum zur\"uckgibt, gilt
    \[
        \pi(S) \leq \sum_{ij} r_j s_{ij} = \sum_j r_j \pi(R_j)
    \]
\end{frame}

\begin{frame}[t]{Beweis konvexe Rechtecksma\ss{}e 2}
    Sei nun $\mu$ ein beliebiges Konvexit\"atsma\ss{}. Dann gilt
    \begin{align*}
        \mu(S) &\leq \sum_i r_i \mu(M_i) &\text{ Konvexit\"at}\\
        &\leq \sum_{i} r_i &\text{ Normalit\"at}\\
        &= \pi(S)
    \end{align*}
\end{frame}
%%%%%%%%%%%%%%%%%%%%%%%%%%%%%%%%%%%%%%%%%%%%%%%%%%%%%%%%%%%%%%%%%%%%%%%%%%%%%%%%%%%%%%%%%%%%%%%%
\section{Konvexe Schranken}
\begin{frame}{Konvexe Schranken}
    \begin{block}{[Hrubes, Jukna, Kulikov, Pudlak 2010]}
        Sei $\mu$ ein Konvexit\"atsma\ss{}, dann gilt
        \[
            \mu(S) \leq \frac{9}{8} (n^2+n)
        \]
        f\"ur jedes $n$-dimensionales Rechteck $S$
    \end{block}
\end{frame}

\begin{frame}
    \begin{block}{Definition zyklische Gruppe}
        Sei $\mathbb{Z}_2 = (\{0,1\}, +, \cdot)$
        der zyklische K\"orper vom Grad 2. \\
        \begin{itemize}
            \item + ist die Addition modulo 2
            \item $\cdot$ ist $\wedge$
        \end{itemize}
    \end{block}
    \pause
    \begin{block}{Satz 1}
        Jeder Vektor $v \in \mathbb{Z}_2^n \backslash \{0\}$ ist genau zu der H\"alfte aller Vektoren aus $\{0,1\}^n$ orthogonal.
    \end{block}
\end{frame}

\begin{frame}[t]{Beweis}
    Seien
    \begin{itemize}
        \item $v,w \in \{0,1\}^n$ beliebig
        \item $s$ Anzahl der Nullen in $v$
        \item $r = n-s$ Anzahl der Einsen in $v$
    \end{itemize}
    \pause
    Orthogonalit\"at ist definiert als
    \[
        \langle v,w \rangle := \sum_{i=1}^n v_i w_i = 0
    \]
    \pause
    Wir haben
    \begin{itemize}
        \item $s$ Freiheitsgerade, da $0 \cdot w_i = 0$
        \item $r-1$ Freiheitsgerade, da $\forall k : 2k \text{ mod } 2 = 0$
    \end{itemize}
    \pause
    Somit
    \[
        2^s \cdot 2^{r-1} = 2^{s+r-1} = 2^{n-1}
    \]
    orthogonale Vektoren
\end{frame}

\begin{frame}
    \begin{block}{Definition Parit\"atsrechtecke}
        Sei $I \subseteq [n] := \{1,\dots,n\}$.\\
        Dann sind die $I$-Parit\"atsrechtecke definiert als
        \begin{align*}
            S_I &:= \{ x \in \{0,1\}^n : \bigoplus_{i \in I} x_i = 0 \}
            \times \{ y \in \{0,1\}^n : \bigoplus_{i \in I} y_i = 1 \}\\
            T_I &= \{(y,x) : (x,y) \in S_I\}
        \end{align*}
    \end{block}
\end{frame}

\begin{frame}{}
    \begin{block}{Satz 2}
        Jede Kante $(x,y) \in \{0,1\}^n \times \{0,1\}^n$, $x\neq y$ geh\"ort zu $2^{n-1}$ Parit\"atsrechtecken.
    \end{block}
    \pause
    \begin{block}{Beweisskizze}
        \begin{itemize}
            \item Sei $I \subseteq [n]$, $v_I \in \{0,1\}^n$ charakteristischer Vektor
        \end{itemize}
        \pause
        \begin{itemize}
            \item $x \oplus y$ nicht orthogonal zu $2^{n-1}$ (Satz 1)
        \end{itemize}
        \pause
        \begin{itemize}
            \item Diese kann man als $v_I$ interpretieren
        \end{itemize}
        \pause
        \begin{itemize}
            \item $(x,y)$ in $2^{n-1}$ Parit\"atsrechtecken
        \end{itemize}
    \end{block}
\end{frame}

\begin{frame}
    \begin{block}{Satz 3}
        Sei $\mu$ ein Rechtecksma\ss{}. Dann gilt f\"ur alle $I \subseteq [n]$
        \[
            \mu(S_I) \leq \frac{9}{8} |I|^2
        \]
        sowie
        \[
            \mu(T_I) \leq \frac{9}{8} |I|^2
        \]
    \end{block}
    \pause
    Folgerung aus
    % TODO: Falls Nachfragen kommen:
    % Wie ich bereits angeschnitten habe, kann man die Rechtecke als Paritaetsrechtecke betrachten.
    % Somit haben wir fuer die Formelgroesse eine Schranke
    % Dann kann man ganz einfach zeigen, dass die Formelgroesse groesser als jedes Rechtecksmass ist,
    % es somit eine obere Schranke bildet.
     \begin{block}{Theorem [Yablonskii 1954]}
        Sei $n \in \mathbb{N}$ mit $n \geq 2$. Dann
        \[
            L(\oplus_n) \leq 3np_n - 2p_n^2 \leq \frac{9}{8}n^2,
        \]
        wobei $p_n = 2^{\lfloor log_2n\rfloor}$.
    \end{block}
\end{frame}

\begin{frame}{Zu Zeigen:}
    \begin{block}{[Hrubes, Jukna, Kulikov, Pudlak 2010]}
        Sei $\mu$ ein Konvexit\"atsma\ss{}, dann gilt
        \[
            \mu(S) \leq \frac{9}{8} (n^2+n)
        \]
        f\"ur jedes $n$-dimensionales Rechteck $S$
    \end{block}
\end{frame}

\begin{frame}[t]{Beweis}
    Seien
    \begin{itemize}
        \item $S$ ein $n$-dimensionales Rechteck, $I \subseteq [n]$, $|I| = i$
    \end{itemize}
    \pause
    \begin{itemize}
        \item $\mu$ Konvexit\"atsma\ss{}
    \end{itemize}
    \pause
    \begin{itemize}
        \item $\mathcal{R}(S)$ die Multimenge aller Parit\"atsrechtecke bzgl $S$ und $I$, sprich
        \[
            (S_I \cup T_I) \cap S
        \]
    \end{itemize}
    \pause
    Dann
    \begin{itemize}
        \item Jede Kante in $S$ geh\"ort zu $2^{n-1}$ Elementen in $\mathcal{R}(S)$
    \end{itemize}
    \pause
    \begin{block}{Satz 2}
        Jede Kante $(x,y) \in \{0,1\}^n \times \{0,1\}^n$, $x\neq y$ geh\"ort zu $2^{n-1}$ Parit\"atsrechtecken.
    \end{block}
\end{frame}

\begin{frame}[t]
    \begin{itemize}
        \item Somit ist $\mathcal{R}(S)$ eine gebrochene Partition mit Gewicht
        \[
             r_R = \frac{1}{2^{n-1}}
        \]
        f\"ur jedes Rechteck $R \in \mathcal{R}(S)$, da
    \end{itemize}
    \pause
    \begin{block}{Gebrochene Partitionen}
        $R_1,\dots,R_m$ bilden eine \textit{gebrochene Partition}, wenn f\"ur alle Kanten $e$
        \[
            \sum_{i : e \in R_i} r_i = 1
        \]
        gilt.
        \end{block}
\end{frame}

\begin{frame}[t,noframenumbering]
    \begin{itemize}
        \item Somit ist $\mathcal{R}(S)$ eine gebrochene Partition mit Gewicht
        \[
             r_R = \frac{1}{2^{n-1}}
        \]
        f\"ur jedes Rechteck $R \in \mathcal{R}(S)$, da
    \end{itemize}
    \begin{itemize}
        \item Zudem wissen wir, dass
        \[
            \mu(R) \leq \frac{9}{8} |i|^2
        \]
    \end{itemize}
\end{frame}

\begin{frame}[t]{}
    Somit folgt
    \begin{align*}
        \mu(S) &\underbrace{\leq}_{\text{konvex}} \sum_{R \in \mathcal{R}(S)} r_R \mu(R) = \frac{1}{2^{n-1}} \sum_{R \in \mathcal{R}(S)} \mu(R)
    \end{align*}
\end{frame}

\begin{frame}[t, noframenumbering]{}
    Somit folgt
    \begin{align*}
        \mu(S) &\leq \sum_{R \in \mathcal{R}(S)} r_R \mu(R)= \frac{1}{2^{n-1}} \sum_{R \in \mathcal{R}(S)} \mu(R)\\
        &= \frac{1}{2^{n-1}} \sum_{i=1}^n \sum_{\substack{R \in \mathcal{R}(S)\\|R| = i}} \mu(R)
    \end{align*}
\end{frame}

\begin{frame}[t, noframenumbering]{}
    Somit folgt
    \begin{align*}
        \mu(S) &\leq \sum_{R \in \mathcal{R}(S)} r_R \mu(R)= \frac{1}{2^{n-1}} \sum_{R \in \mathcal{R}(S)} \mu(R)\\
        &= \frac{1}{2^{n-1}} \sum_{i=1}^n \colorbox{dominant_blue}{$\strut\displaystyle\sum\limits_{{\substack{R \in \mathcal{R}(S)\\|R| = i}}}$} \colorbox{roman_silver}{$\strut\mu(R)$}\\
        &\leq \frac{1}{2^{n-1}} \sum_{i=1}^n 2 \colorbox{dominant_blue}{$\displaystyle\binom{n}{i}$} \colorbox{roman_silver}{$\strut\dfrac{9}{8} i^2$} 
    \end{align*}
    wobei
    \begin{itemize}
        \item \colorbox{dominant_blue}{$\strut|I| = i$}
        \item \colorbox{roman_silver}{$\strut\displaystyle\mu(R) \leq \frac{9}{8} i^2$} (Satz 3)
        \item $2$, da $S_I$ und $T_I$
    \end{itemize}
\end{frame}

\begin{frame}[t, noframenumbering]{}
    Somit folgt
    \begin{align*}
        \mu(S) &\leq \sum_{R \in \mathcal{R}(S)} r_R \mu(R)= \frac{1}{2^{n-1}} \sum_{R \in \mathcal{R}(S)} \mu(R)\\
        &= \frac{1}{2^{n-1}} \sum_{i=1}^n \sum_{\substack{R \in \mathcal{R}(S)\\|R| = i}} \strut\mu(R)\\
        &\leq \frac{1}{2^{n-1}} \sum_{i=1}^n 2 \binom{n}{i} \strut\dfrac{9}{8} i^2 = \frac{1}{2^{n-2}} \frac{9}{8} \sum_{i=1}^n \binom{n}{i} i^2
    \end{align*}
\end{frame}

\begin{frame}[t, noframenumbering]{}
    Somit folgt
    \begin{align*}
        \mu(S) &\leq \sum_{R \in \mathcal{R}(S)} r_R \mu(R)= \frac{1}{2^{n-1}} \sum_{R \in \mathcal{R}(S)} \mu(R)\\
        &= \frac{1}{2^{n-1}} \sum_{i=1}^n \sum_{\substack{R \in \mathcal{R}(S)\\|R| = i}} \strut\mu(R)\\
        &\leq \frac{1}{2^{n-1}} \sum_{i=1}^n 2 \binom{n}{i} \strut\dfrac{9}{8} i^2 = \frac{1}{2^{n-2}} \frac{9}{8} \sum_{i=1}^n \binom{n}{i} i^2\\
        &= \dots
    \end{align*}
\end{frame}

\begin{frame}[t, noframenumbering]{}
    Somit folgt
    \begin{align*}
        \mu(S) &\leq \sum_{R \in \mathcal{R}(S)} r_R \mu(R)= \frac{1}{2^{n-1}} \sum_{R \in \mathcal{R}(S)} \mu(R)\\
        &= \frac{1}{2^{n-1}} \sum_{i=1}^n \sum_{\substack{R \in \mathcal{R}(S)\\|R| = i}} \strut\mu(R)\\
        &\leq \frac{1}{2^{n-1}} \sum_{i=1}^n 2 \binom{n}{i} \strut\dfrac{9}{8} i^2 = \frac{1}{2^{n-2}} \frac{9}{8} \sum_{i=1}^n \binom{n}{i} i^2\\
        &= \dots\\
        &= \frac{1}{2^{n-2}} \frac{9}{8} (n^2 + n) 2^{n-2}
    \end{align*}
\end{frame}

\begin{frame}[t, noframenumbering]{}
    Somit folgt
    \begin{align*}
        \mu(S) &\leq \sum_{R \in \mathcal{R}(S)} r_R \mu(R)= \frac{1}{2^{n-1}} \sum_{R \in \mathcal{R}(S)} \mu(R)\\
        &= \frac{1}{2^{n-1}} \sum_{i=1}^n \sum_{\substack{R \in \mathcal{R}(S)\\|R| = i}} \strut\mu(R)\\
        &\leq \frac{1}{2^{n-1}} \sum_{i=1}^n 2 \binom{n}{i} \strut\dfrac{9}{8} i^2 = \frac{1}{2^{n-2}} \frac{9}{8} \sum_{i=1}^n \binom{n}{i} i^2\\
        &= \dots\\
        &= \frac{1}{2^{n-2}} \frac{9}{8} (n^2 + n) 2^{n-2} \\
        &= \frac{9}{8} (n^2+n)
    \end{align*}
\end{frame}

%%%%%%%%%%%%%%%%%%%%%%%%%%%%%%%%%%%%%%%%%%%%%%%%%%%%%%%%%%%%%%%%%%%%%%%%%%%%%%%%%%%%%%%%%%%%%%%%

\begin{frame}{Zusammenfassend:}
    \begin{itemize}
        \item Konvexe Ma\ss{}e k\"onnen nicht die $O(n^2)$ Schranke brechen
    \end{itemize}
    \pause
    \begin{block}{[Hrubes, Jukna, Kulikov, Pudlak 2010]}
        Sei $\mu$ ein Konvexit\"atsma\ss{}, dann gilt
        \[
            \mu(S) \leq \frac{9}{8} (n^2+n)
        \]
        f\"ur jedes $n$-dimensionales Rechteck $S$
    \end{block}
    \pause
    \begin{itemize}
        \item Danke f\"ur alle tollen Vortr\"age
    \end{itemize}
\end{frame}