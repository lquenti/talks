\documentclass{beamer}
\title{What Every Programmer Should Know About Memory}
\author{Lars Quentin}
\date{09.02.2022}
\usetheme{Copenhagen}

\begin{document}

% - Willkommen
% - Reden ueber das Paper "What Every Programm..."
%   - 2007 von Ulrich Drepper bei RedHat geschrieben
% - Name ist Luege, da es viel zu sehr ins Detail geht
%   - Wir beleuchten hier nur einen Bruchteil
% - Ueberleitung: Fangen wir mit ein paar Definitionen an:
\frame{\titlepage}


% Folie 1:
% - Okay fangen wir einfach mal mit der Definition eines CPU caches an.
% - <Vorlesen>
% PAUSE
% - War nur ein Scherz!
% - Das hier ist ein Talk, keine trockene Vorlesung
% - Um alle auf den selben Stand zu bringen, lasst uns bei absolut 0 anfangen.
% - Und lasst uns dieses haessliche Theme weglassen.
\begin{frame}[t]
\frametitle{Definitionen}
Ein CPU-Cache ist ein 6 Tuple $(n, l_{size}, assoc, wp, srp)$ wobei
~\\~\\~\\~\\~\\~\\~\\
{\Huge
~~~~~~~~~~~~bla bla bla
}
\end{frame}

\end{document}