% spaces
\newcommand{\K}{\mathbb{K}}
\newcommand{\N}{\mathbb{N}}
\newcommand{\Z}{\mathbb{Z}}
\newcommand{\Q}{\mathbb{Q}}
\newcommand{\R}{\mathbb{R}}
\newcommand{\C}{\mathbb{C}}

% logic
\newcommand{\larr}{\leftarrow}
\newcommand{\rarr}{\rightarrow}
\newcommand{\Larr}{\Leftarrow}
\newcommand{\Rarr}{\Rightarrow}
\newcommand{\lrarr}{\leftrightarrow}
\newcommand{\Lrarr}{\Leftrightarrow}

% Landau
\newcommand{\BigO}{\mathcal{O}}

% norm
\newcommand{\norm}[2]{\left\lVert{#1}\right\rVert_{#2}}

% vectors
% used as
% \colvec{5}{a}{b}{c}{d}{e}
% stolen from https://tex.stackexchange.com/a/2712
\newcount\colveccount
\newcommand*\colvec[1]{
        \global\colveccount#1
        \begin{pmatrix}
        \colvecnext
}
\def\colvecnext#1{
        #1
        \global\advance\colveccount-1
        \ifnum\colveccount>0
                \\
                \expandafter\colvecnext
        \else
                \end{pmatrix}
        \fi
}

% better greek
\renewcommand*{\epsilon}{\varepsilon}
\renewcommand*{\phi}{\varphi}
\renewcommand*{\rho}{\varrho}

% Zu zeigen
\newcommand{\zz}{$ \mathrm{Z\kern-.3em\raise-0.5ex\hbox{Z}}$}

% Theoreme
% (definition theoremstyle sets the text upright)
\theoremstyle{definition}\newtheorem*{definition}{Definition}
\theoremstyle{definition}\newtheorem*{beispiel}{Beispiel}
\theoremstyle{definition}\newtheorem*{notation}{Notation}
\theoremstyle{definition}\newtheorem*{bemerkung}{Bemerkung}
\theoremstyle{definition}\newtheorem*{uebung}{\"Ubung}
\theoremstyle{definition}\newtheorem*{idee}{Idee}
\theoremstyle{definition}\newtheorem*{note}{Note}

\newtheorem{satz}{Satz}
