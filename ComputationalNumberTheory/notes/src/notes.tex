\documentclass{article}
\usepackage[utf8]{inputenc}

\title{Notes for Seminar "Computational Number Theory"}
\date{\today}

% Math Packages
\usepackage{amsmath,amsthm,amssymb,amsfonts}

% More advanced <ol>'s
\usepackage{enumerate}

% Simple Code formatting, no syntax highlighting
\usepackage{listings}

% PDF-Integration
\usepackage{pdfpages}

% \mathclap
% See: https://pbelmans.ncag.info/blog/2010/12/06/the-power-of-mathclap-and-substack/
\usepackage{mathtools}

% Graphic Packages
% Also, we use the "float" package for \begin{figure}[H].
% The H-Parameter is described as follows:
%
% Places the float at precisely the location in the LaTeX code. Requires the `float` package, though it may cause
% problems occasionally. This is somewhat equivalent to `h!`.
%
% See: https://de.overleaf.com/learn/latex/Inserting_Images
\usepackage{graphicx}
\usepackage{float}

% spaces
\newcommand{\K}{\mathbb{K}}
\newcommand{\N}{\mathbb{N}}
\newcommand{\Z}{\mathbb{Z}}
\newcommand{\Q}{\mathbb{Q}}
\newcommand{\R}{\mathbb{R}}
\newcommand{\C}{\mathbb{C}}

% logic
\newcommand{\larr}{\leftarrow}
\newcommand{\rarr}{\rightarrow}
\newcommand{\Larr}{\Leftarrow}
\newcommand{\Rarr}{\Rightarrow}
\newcommand{\lrarr}{\leftrightarrow}
\newcommand{\Lrarr}{\Leftrightarrow}

% Landau
\newcommand{\BigO}{\mathcal{O}}

% norm
\newcommand{\norm}[2]{\left\lVert{#1}\right\rVert_{#2}}

% vectors
% used as
% \colvec{5}{a}{b}{c}{d}{e}
% stolen from https://tex.stackexchange.com/a/2712
\newcount\colveccount
\newcommand*\colvec[1]{
        \global\colveccount#1
        \begin{pmatrix}
        \colvecnext
}
\def\colvecnext#1{
        #1
        \global\advance\colveccount-1
        \ifnum\colveccount>0
                \\
                \expandafter\colvecnext
        \else
                \end{pmatrix}
        \fi
}

% better greek
\renewcommand*{\epsilon}{\varepsilon}
\renewcommand*{\phi}{\varphi}
\renewcommand*{\rho}{\varrho}

% Zu zeigen
\newcommand{\zz}{$ \mathrm{Z\kern-.3em\raise-0.5ex\hbox{Z}}$}

% Theoreme
% (definition theoremstyle sets the text upright)
\theoremstyle{definition}\newtheorem*{definition}{Definition}
\theoremstyle{definition}\newtheorem*{beispiel}{Beispiel}
\theoremstyle{definition}\newtheorem*{notation}{Notation}
\theoremstyle{definition}\newtheorem*{bemerkung}{Bemerkung}
\theoremstyle{definition}\newtheorem*{uebung}{\"Ubung}
\theoremstyle{definition}\newtheorem*{idee}{Idee}
\theoremstyle{definition}\newtheorem*{note}{Note}

\newtheorem{satz}{Satz}

% suppress date if not defined otherwise
\date{}

% I am the author if not defined otherwise
\author{Lars Quentin}


\usepackage{graphicx}
\usepackage{hyperref}
\usepackage{amsmath, amsthm, amsfonts, amssymb}
\usepackage{float}
\usepackage[a4paper, total={6in, 8in}]{geometry}

% TODO: Make visualisation for why the discrete convolution represents the multiplication of 2 polynomials
% TODO: Test everything polynomial also with big numbers
%         - Is the connection that the x^i are like 10^i in the basis deconstruction?
% Use "Order" for telling Big-O runtimes


\begin{document}

\maketitle
\section{Notes "Modern Computer Algebra"}
\subsection{Fundamental Algorithms}
\subsubsection{Multiplication}
Following our program, we first consider the product $c = a \cdot b$ of two polynomials $a,b \in R[x]$. Its
coefficients are
\[
	c_k = \sum_{\substack{0 \leq i \leq n\\0 \leq j \leq m\\i+j=k}} a_ib_j
\]
for $0 \leq k \leq n+m$.\\
\underline{Algorithm 1}:\\
Input: The coefficients of $a = \sum_{0 \leq i \leq n} a_ix^i$ and $b=\sum_{0 \leq i \leq m} b_i x^i$ in $R[x]$,
where $R$ is a (commutative) ring.\\
Output: The coefficients of $c = a \cdot b \in R[x]$
\begin{verbatim}
1. for i=0,...,n do d_i <- a_i x^i * b
2. return c = sum(d_i)
\end{verbatim}
To understand the complexity let's look at an example:
\begin{align*}
	&(a_4 x^4 + a_3 x^3 + a_2 x^2 + a_1 x^1)(b_4 x^4 + b_3 x^3 + b_2 x^2 + b_1 x^1)\\
	=& \sum_{i=1}^4 a_i x^i \cdot (b_4 x^4 + b_3 x^3 + b_2 x^2 + b_1 x^1)\\
	=& \sum_{i=1}^4 a_i x^i b_4 x^4 + a_i x_i b_3 x^3 + a_i x_i b_2 x^2 + a_i x_i a_1 x^1
\end{align*}
We need $n^2$ multiplications since we multiply everything by everything.\\
We need $(n-1)$ additions per iteration, thus $(n-1)\cdot n \approx n^2$ in total.\\
This is because of the following:\\
The multiplication of $a_ix^i \cdot b$ is realized as the multiplication of each $b_j$ by $a_i$ plus a shift by $i$
places. The variable $x$ serves us just as a convinient way to write polynomials, and there is no arithmetic involved
in "multiplying" by $x$ or any power of it.\\
See the formal definition of polynomial rings for more intuition.
\subsection{Fast Multiplication}
\subsubsection{Karatsuba's multiplication algorithm}
We start with the multiplication of two polynomials $f,g \in R[x]$ of degree less than $n$ over a Ring $R$.
As usual, "ring" means a commutative ring with $1$.
\begin{definition}
	A \emph{ring} $(R, +, \cdot)$ is a set $R$ with 2 binary operations $+$ and $\cdot$ which satisfy the
	following axioms.
	\begin{enumerate}
		\item $(R,+)$ is an abelian group under addition, meaning that:
		\begin{itemize}
			\item $+$ is associative.
			\item $+$ is commutative.
			\item There is an neutral element $0 \in R$.
			\item Every element as an additive inverse.
		\end{itemize}
	\item $(R, \cdot)$ is a semigroup under multiplication, meaning that $\cdot$ is associative.
	\item Multiplication is distributive with respect to addition.
	\end{enumerate}
\end{definition}
\begin{definition}
	An \emph{commutative ring with $1$} $(R, +, \cdot)$ is a ring that satisfies the following:
	\begin{enumerate}
		\item $(R,\cdot)$ is not only a semigroup but also a monoid, meaning that:
		\begin{itemize}
			\item $(R, \cdot)$ is associative (i.e. a semigroup).
			\item There exists a neutral element $1 \in R$ over multiplication (i.e. the
			multiplicative identity).
		\end{itemize}
		\item $(R, \cdot)$ is commutative.
	\end{enumerate}
	A ring with $1$ is also called a unitary ring.
\end{definition}
\begin{definition}
	Let $D \subseteq \Z, f,g : D \rarr \mathbb{C}$.\\
	A \emph{discrete convolution} of $f$ and $g$ is defined as
	\[
		(f * g)(n) = \sum_{m=-\infty}^\infty f(m) g(n-m)
	\]
	When $g$ has finite support (i.e. has finitely many elements in the domain producing a nonzero value)
	over the set $\{-M, -M+1, \dots, M\}$ then it can be simplified to
	\[
		(f * g)(n) = \sum_{m = -M}^M f(n-m)g(m)
	\]
\end{definition}
\begin{definition}
	A \emph{polynomial ring} $R[X]$ is a commutative ring $(R^{(\mathbb{N}_0)}, +, \cdot)$ with 1 defined as
	\begin{itemize}
		\item $R^{(\mathbb{N}_0)}$ is the set
		\[
			R^{(\mathbb{N}_0)} :=
			\{
				(a_i)_{i \in \mathbb{N}_0} | a_i \in R, a_i = 0 \text{ for almost all } i
			\}
		\]
		"Almost all" means that it holds for all but a negligible amount.
		\item $+$ is defined as the compontentwise addition, meaning that
		\[
			(a_i)_{i \in \mathbb{N}_0} + (b_i)_{i \in \mathbb{N}_0} := (a_i+b_i)_{i \in \mathbb{N}_0}
		\]
		\item $\cdot$ is defined as the discrete convolution, meaning that
		\[
			(a_i)_{i \in \mathbb{N}_0} \cdot (b_i)_{i \in \mathbb{N}_0}
			:= \left( \sum_{i=0}^k a_i b_{k-i} \right)_{k \in \mathbb{N}_0}
			= \left( \sum_{i+j=k} a_i b_j \right)_{k \in \mathbb{N}_0}
		\]
		Since we have a finite sequences, it is also called the \emph{Cauchy Product}
		\item Let $X$ be defined such that
		\begin{itemize}
			\item $X \in R^{(\mathbb{N}_0)}$ is defined as
			\[
				X = X^1 := (0,1,0,0,\dots)
			\]
			\item $1 \in R^{(\mathbb{N}_0)}$ is defined as
			\[
				1 := X^0 = (1,0,0,\dots)
			\]
			\item Every power $X^k \in R^{(\mathbb{N}_0)}$, $k \in \mathbb{N}_0$ is defined as
			\[
				X^k := \underbrace{X \cdot X \cdot \dots \cdot X}_{k \text{ times}}
				= (\underbrace{0,0,\dots,0}_{k \text{ zeros}},1,0,0,\dots)
			\]
		\end{itemize}
		With $X$ defined we can write any polynomial $f \in R[X]$ as
		\[
			f = \sum_{i=0}^n a_i X_i
		\]
		where $n$ is the largest non-zero index.
	\end{itemize}
\end{definition}
\begin{note}
	Let $p,q \in R[X]$ be 2 polynomials. When written in polynomial notation the multiplication is equivalent
	to the intuitive multiplication of both terms, i.e.
	\[
		p+q = s_0 + s_1 \cdot X + \dots + s_{(\deg(p)+\deg(q)} X^{(\deg(p)+\deg(q))}
	\]
	where
	\begin{itemize}
		\item $s_i = p_0q_i + p_1 q_{i-1} + \dots + p_i q_0$
		\item $\deg(p)$ notes the degree of $p$, i.e. it's largest exponent
	\end{itemize}
\end{note}
If $f_i, g_j, h_k \in R$ are the coefficients of $f,g$ and $h=fg$, respectively, the classical multiplication
algorithm uses $O(n^2)$ operations in $R$ to compute the $h_k$ from the $f_i$ and $g_j$:\\
$n^2$ multiplications $f_i g_j$ plus $(n-1)^2$ additions for all $h_k = \sum_{i+j=k} f_i g_j$.\\
For instance, multiplying $(ax+b)(cx+d) = acx^2 + (ad+bc)x+bd$ uses four multiplications $ac$, $ad$, $bc$, $bd$.\\

Suprisingly, there is an easy method of doing better. We compute $ac$, $bd$, $u=(a+b)+(c+d)$ and $ad+bc=u-ac-bd$,
with three multiplications and four additions and subtractions. The total has increased to seven operations, but
a recusrive application will drastically recuce the overall cost (See Figure 8.2).
% TODO: Figure 8.2 analysieren
To explain the general approach, we assume that $n=2^k$ for some $k \in \mathbb{N}$ (otherwise we can just fill
up with zeros), set $m = n/2$, and rewrite
$f$ and $g$ in the form $f = F_1x^m + F_0$ with $F_0, F_1 \in R[x]$ of degree less than $m$ and similarly
$g = G_1 x^m + G_0$. (If $\deg(f) < n+1$, then some of the top coefficients are zero.)
Now $fg = F_1G_1 x^n + (F_0G_1+F_1G_0) + F_0G_0$. In this form, multiplication of $f$ and $g$ has been reduced
to four multiplications of polynomials of degree less than $m$. Multiplication by a power of $x$ does not count
as a multiplication, since it corresponds merely to a shift of the coefficients.\\

So far we have not really archieved anything. But the method by Karatsuba in Karatsuba \& Ofman (1962), explained
for $n=1$ above, shows how this expression for $fg$ can be rearranged to reduce the number of multiplications
of the smaller polynomials at the expense of increasing the number of additions. Since multiplication is slower
than addition, a saving is obtained when $n$ is sufficiently large. We rewrite this product as
\[
	fg = F_1 G_1 x^n + ((F_0 + F_1)(G_0+G_1) - F_0 G_0 - F_1 G_1) x^m + F_0 G_0
\]
This expression shows that multiplication of $f$ and $g$ requires only three multiplications of polynomials of
degree less than $m$ and some additions. The same method is now applied recursively to the smaller
multiplications. If $T(n)$ denotes the time necessary to multiply two polynomials of degree less than $n$, then
\[
	T(2n) \leq 3T(n) + cn
\]
for some constant $c$. The linear term comes from the observation that addition of two polynomials of degree
less than $d$ can be done with $d$ operations in $R$.%\\
\end{document}

